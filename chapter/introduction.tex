	\begin{frame}[plain,c]
		\begin{center}
			\huge Introduction
		\end{center}
	\end{frame}


	\begin{frame}
		\frametitle{Outline}
		\begin{itemize}
			\item Challenge on \textit{combinatorial design}%nato dalla mate ricreativa
			\item What is it?:
			\begin{itemize}
				\item existence or non-existence
				\item representation
				\item construction
			\end{itemize}
		\end{itemize}
	\end{frame}

	%definizione di sts
	%meglio formalismo e da qui si vede il combinatorio e la definizione operativa
	%attenzione che sono insiemi (il codominio è indistinguile (forse))
	%Steiner triple systems were apparently defined for the first time by W. S. B. Wool-house [35] (Prize question 1733, Lady’s and Gentlemens’ Diary, 1844)
	\begin{frame}
		\frametitle{What is Steiner Triple System }
		\begin{block}{(Definition) Steiner Triple Systems (STS)}
			is an ordered pair $(S,T)$ (a \textit{design}) where $S$ is a finite set of \textit{point}/\textit{symbol} and $T$ is a set of subsets of 3-symbol in which all possible pair of $S$ are contained \textbf{once and only once}.\\
			 
		\end{block}
	
	\pause
	
	%pensiamo al bruteforce e vediamo che complessità abbiamo
	More formally:\\
	\begin{itemize}
		\item define $S$ such that $|S|=v$
		\item 	then $T = \{ \forall \{a,b,c\} \in S\times S \times S\}$\\
		 such that $\forall a,b \in S \times S \ a \not = b  \quad \sum_{ \forall \{x,y,z\} \in T} (\mathbb{I}_{ \{ a,b \} \in \{x,y\} }  + \mathbb{I}_{ \{ a,b \} \in \{y,z\} } +\mathbb{I}_{ \{ a,b \} \in \{z,x\}  }) = 1 $
	\end{itemize}
	More compact way to define STS by define the \textit{order} $v$ of STS by $v = |S|$
	\end{frame}

	%possiamo parlare di famiglia parlando di STS(v)
	
	
	%vediamo qualche esempio
	%prima cosa: devo avere T formato da triplette di elementi distinti, quindi ordine 1 2 non esiste sts.
	%per il 4 5 6 è qualcosa di intrinseco, non si vede dai numeri a colpo d'occhio
	%per il 4 è semplice coonteggiare e creare le coppie di elementi distinti (6). Si pensa quindi di facilmente creare 2 set da 3 elementi ma se si prova ci si accorge che o abbiamo un set da 4 elementi distinti e uno da 3 oppure dobbiamo ripetere alcune coppie per ottenere un gruppo da  3 (ma non valido perchè vengono ripetute le coppie)
	\begin{frame}
		\frametitle{Examples of STS}
		\begin{center}
			$S=\{a\}, T= \emptyset$\\
			$S=\{a,b\}, T= \emptyset$\\
			\pause
			$S=\{a,b,c\}, T= \{ \{a,b,c\} \}$\\
			\pause
			$S=\{a,b,c,d\}, T=\emptyset$\\
			$S=\{a,b,c,d,e\}, T= \emptyset$\\
			$S=\{a,b,c,d,e,f\}, T= \emptyset$\\
			\pause
			$S=\{a,b,c,d,e,f,g\}, T= \{ \{a,b,c\},\{c,d,e\},\{e,f,a\},\{f,b,d\},\{a,g,d\},\{e,g,b\},\{c,g,f\} \}$\\
			...
		\end{center}
	\end{frame}

	\begin{frame}
		\frametitle{Balanced incomplete blocks design}
		\begin{block}{(Definition) $(v,k,\lambda)-\mathrm{BIBD}$}
			$v$,$k$ and $\lambda$ be positive integers such that $v > k \ge 2$. A balanced incomplete block design is a \textit{design} $(S,T)$ such that satisfy these properties:
			\begin{enumerate}
				\item $|S|=v$
				\item $\forall t \in T\quad |t|= k $
				\item for all distinct pairs are contained in exactly $\lambda$ blocks ($t$) 
			\end{enumerate}
		\end{block}
	

		Why \textbf{balanced} and \textbf{incomplete}?\\
		\begin{description}[l]
			\item[balanced] they share the same property (2)
			\item[incomplete] by reason of $v=|S|\quad > \quad k=|t| \  \forall t \in T$
		\end{description}

	\end{frame}

	\begin{frame}
	\frametitle{What is Steiner Triple System 2}
	
	$\lambda$ blocks ($t$) of $(v,k,\lambda)-\mathrm{BIBD}$ iff \textcolor{red}{$\lambda = 1$, $k=3$}.
	
	\begin{block}{$(v,k,\lambda)-\mathrm{BIBD}$}
		$v$,$k$ and $\lambda$ be positive integers such that $v > k \ge 2$. A balanced incomplete block design is a \textit{design} $(S,T)$ such that satisfy these properties:
		\begin{enumerate}
			\item $|S|=v$
			\item $\forall t \in T\quad |t|= k $
			\item $\forall s \in S$ is contained in exactly $\lambda$ blocks ($t$) 
		\end{enumerate}
	\end{block}
	\pause
	%I'm not sure 10000000% 
	\textcolor{red}{All theory from BIBD is shared too in STS}
	
	%ma qual'è la relazione tra essi ?
	% If a (v, k, λ)-BIBD exists, then λ(v − 1) ≡ 0 (mod k − 1) and
%	λv(v − 1) ≡ 0 (mod k(k − 1))  non riusciamo ad avere qualcosa di più forte ? 
	\end{frame}


%kirkman nel 1847 ma steiner nel 53 
\begin{frame}
	\frametitle{[Kirkman, 1847]Existence proof}
	\begin{theorem}
		A STS of \textit{order} $v$ \textbf{exists} if and only if $v\equiv 1,3\ mod(6)$  
	\end{theorem}
	\begin{proof}
		($\Rightarrow$)We know that all possible pairs are $\binom{v}{2}$, and by definition these pairs are partitioned (non-overlapping and union make all) into 3-element groups. Thoose groups are $|T|=\frac{\binom{v}{2}}{3}=\frac{v(v-1)}{6}$. Then for $\forall x \in S$ can be defined $T(x) = \{ t\not\{x\} | \textcolor{red}{x \in t} \in T \}$. So if an $x \in S$ is fixed and then for every set $t$ which contain $x$ we remove the point $x$ then we carry out $v-1$ point partitioned in \textit{2-element} set. As we can't make 2-element partition from a group of \textit{odd} element, $v-1$ is \textit{even}! So $v$ is \textit{odd} and it's equal to say $v \equiv 1,3,5 mod(6)$. The $\frac{v(v-1)}{6}$ is not an integer for every $v \equiv 5 mod(6)$. As a result STS $\Rightarrow$ $v\equiv 1,3\ mod(6)$
	\end{proof}
\end{frame}

\begin{frame}
\frametitle{Existence proof 2}
\begin{block}{$(S,T)\ : |S|=v\ \wedge v\equiv 1,3\ mod(6) \Rightarrow STS(v)$}
	In addition we suppose:
	\begin{itemize}
		\item each dinstict pair of $S$ belongs to \textit{at least} one triple in $T$
		\item $|T| \le \frac{v(v-1)}{6}$
	\end{itemize}
\end{block}
\begin{proof}[Proof 1]
	(Absurd) Assume the contrary and make a list $L$ as follows: for every pair write down the triple with which it is associated. Then $|L|> \binom{v}{2}$ as there exists a pair with towo triples. Now since each triple is counted by exactly three pairs so $|T| =|L|/3 > \frac{\binom{v}{2}}{3}$, a contradiction.
\end{proof}


\end{frame}

\begin{frame}
%oppure
\begin{proof}[Proof 2]
	For each distinct pair of $S$ belongs to at least one triple and if the number of triples is less than or equal to the right number of triples, then each pair of sumbols in $S$ belongs to exactly one triple in $T$.
\end{proof}

%la condizione necessaria e il fatto che sia sufficiente solo per queste due strutture di STS(v) ci permette di dimostrare la suff.
\begin{proof}[Proof 3]
	We costruct 2 methods to prove sufficient costraint to the Theorem by showing 2 methods:
	\begin{itemize}
		\item Bose construction
		\item Skolem construction
	\end{itemize}
\end{proof}
\end{frame}