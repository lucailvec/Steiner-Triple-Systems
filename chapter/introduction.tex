	\begin{frame}[plain,c]
		\begin{center}
			\huge Introduction
		\end{center}
	\end{frame}


	\begin{frame}
		\frametitle{Outline}
		\begin{itemize}
			\item Challenge on \textit{combinatorial design}%nato dalla mate ricreativa
			\item What is it?:
			\begin{itemize}
				\item existence or non-existence
				\item representation
				\item construction
			\end{itemize}
		\end{itemize}
	\end{frame}

	%definizione di sts
	%meglio formalismo e da qui si vede il combinatorio e la definizione operativa
	%attenzione che sono insiemi (il codominio è indistinguile (forse))
	
	\begin{frame}
		\frametitle{What is Steiner Triple System }
		\begin{block}{(Definition) Steiner Triple Systems (STS)}
			is an ordered pair $(S,T)$ (a \textit{design}) where $S$ is a finite set of \textit{point}/\textit{symbol} and $T$ is a set of subsets of 3-symbol in which all possible pair of $S$ are contained \textbf{once and only once}.\\
			 
		\end{block}
	
	\pause
	
	
	More formally:\\
	\begin{itemize}
		\item define $S$ such that $|S|=v$
		\item 	then $T = \{ \forall \{a,b,c\} \in S\times S \times S\}$\\
		 such that $\forall a,b \in S \times S \ a \not = b  \quad \sum_{ \forall \{x,y,z\} \in T} (\mathbb{I}_{ \{ a,b \} \in \{x,y\} \wedge \{y,z\} \wedge \{z,x\} }) = 1 $
	\end{itemize}
	More compact way to define STS by define the \textit{order} $v$ of STS by $v = |S|$
	\end{frame}

	%possiamo parlare di famiglia parlando di STS(v)
	
	
	%vediamo qualche esempio
	
	\begin{frame}
		\frametitle{Examples of STS}
		\begin{center}
			$S=\{a\}, T= \emptyset$\\
			$S=\{a,b\}, T= \emptyset$\\
			$S=\{a,b,c\}, T= \{ \{a,b,c\} \}$\\
			$S=\{a,b,c,d\}, T=\emptyset$\\
			$S=\{a,b,c,d,e\}, T= \emptyset$\\
			$S=\{a,b,c,d,e,f\}, T= \emptyset$\\
			$S=\{a,b,c,d,e,f,g\}, T= \{ \{a,b,c\},\{c,d,e\},\{e,f,a\},\{f,b,d\},\{a,g,d\},\{e,g,b\},\{c,g,f\} \}$\\
			...
		\end{center}
	\end{frame}

	\begin{frame}
		\frametitle{Balanced incomplete blocks design}
		\begin{block}{(Definition) $(v,k,\lambda)-\mathrm{BIBD}$}
			$v$,$k$ and $\lambda$ be positive integers such that $v > k \ge 2$. A balanced incomplete block design is a \textit{design} $(S,T)$ such that satisfy these properties:
			\begin{enumerate}
				\item $|S|=v$
				\item $\forall t \in T\quad |t|= k $
				\item for all distinct pairs are contained in exactly $\lambda$ blocks ($t$) 
			\end{enumerate}
		\end{block}
	

		Why \textbf{balanced} and \textbf{incomplete}?\\
		\begin{description}[l]
			\item[balanced] they share the same property (2)
			\item[incomplete] by reason of $v=|S|\quad > \quad k=|t| \  \forall t \in T$
		\end{description}

	\end{frame}

	\begin{frame}
	\frametitle{What is Steiner Triple System 2}
	
	$\lambda$ blocks ($t$) of $(v,k,\lambda)-\mathrm{BIBD}$ iff \textcolor{red}{$\lambda = 1$, $k=3$}.
	
	\begin{block}{$(v,k,\lambda)-\mathrm{BIBD}$}
		$v$,$k$ and $\lambda$ be positive integers such that $v > k \ge 2$. A balanced incomplete block design is a \textit{design} $(S,T)$ such that satisfy these properties:
		\begin{enumerate}
			\item $|S|=v$
			\item $\forall t \in T\quad |t|= k $
			\item $\forall s \in S$ is contained in exactly $\lambda$ blocks ($t$) 
		\end{enumerate}
	\end{block}

	Requirement \textbf{1} satisfied by definition of STS on the design $(S,T)$.\\
	Requirement \textbf{2} satisfied by definition of STS.\\	 
	Requirement \textbf{3} for all distinct pairs are contained in exactly \\
	\pause
	%I'm not sure 10000000% 
	\textcolor{red}{All theory from BIBD is shared too in STS}
	
	\end{frame}
