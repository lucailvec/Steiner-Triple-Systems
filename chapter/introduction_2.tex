%kirkman nel 1847 ma steiner nel 53 
\begin{frame}
	\frametitle{[Kirkman, 1847]Existence proof}
	\begin{theorem}
		A STS of \textit{order} $v$ \textbf{exists} if and only if $v\equiv 1,3\ mod(6)$  
	\end{theorem}
	\begin{proof}
		($\Rightarrow$)We know that all possible pairs are $\binom{v}{2}$, and by definition these pairs are partitioned (non-overlapping and union make all) into 3-element groups. Those groups are $|T|=\frac{\binom{v}{2}}{3}=\frac{v(v-1)}{6}$. Then for $\forall x \in S$ can be defined $T(x) = \{ t\not\{x\} | \textcolor{red}{x \in t} \in T \}$. So if an $x \in S$ is fixed and then for every set $t$ which contain $x$ we remove the point $x$ then we carry out $v-1$ point partitioned in \textit{2-element} set. As we can't make 2-element partition from a group of \textit{odd} element, $v-1$ is \textit{even}! So $v$ is \textit{odd} and it's equal to say $v \equiv 1,3,5 mod(6)$. The $\frac{v(v-1)}{6}$ is not an integer for every $v \equiv 5 mod(6)$. As a result STS $\Rightarrow$ $v\equiv 1,3\ mod(6)$
	\end{proof}
\end{frame}

\begin{frame}
\frametitle{Existence proof 2}
\begin{block}{$(S,T)\ : |S|=v\ \wedge v\equiv 1,3\ mod(6) \Rightarrow STS(v)$}
	In addition we suppose:
	\begin{itemize}
		\item each dinstict pair of $S$ belongs to \textit{at least} one triple in $T$
		\item $|T| \le \frac{v(v-1)}{6}$
	\end{itemize}
\end{block}
\begin{proof}[Proof 1]
	(Absurd) Assume the contrary and make a list $L$ as follows: for every pair write down the triple with which it is associated. Then $|L|> \binom{v}{2}$ as there exists a pair with tow triples. Now since each triple is counted by exactly three pairs so $|T| =|L|/3 > \frac{\binom{v}{2}}{3}$, a contradiction.
\end{proof}


\end{frame}

\begin{frame}
%oppure
\begin{proof}[Proof 2]
	For each distinct pair of $S$ belongs to at least one triple and if the number of triples is less than or equal to the right number of triples, then each pair of sumbols in $S$ belongs to exactly one triple in $T$.
\end{proof}

%la condizione necessaria e il fatto che sia sufficiente solo per queste due strutture di STS(v) ci permette di dimostrare la suff.
\begin{proof}[Proof 3]
	We costruct 2 methods to prove sufficient costraint to the Theorem by showing 2 methods:
	\begin{itemize}
		\item Bose construction
		\item Skolem construction
	\end{itemize}
\end{proof}
\end{frame}
