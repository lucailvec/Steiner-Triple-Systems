\begin{frame}[plain,c]
\begin{center}
	\huge Practical example
\end{center}
\end{frame}

\begin{comment}
Steiner triple systems were apparently defined for the first time by W. S. B. Wool-
house [35] (Prize question 1733, Lady’s and Gentlemens’ Diary, 1844)


In 1853 J. Steiner posed the sufficiency of this theorem as a problem and it was proved in 1859 by M. Reiss. Neither mathematician was aware of the fact that the problem had been posed and solved by T.P. Kirkman in an 1847 article appearing in the Cambridge and Dublin Mathematical Journal. Indeed, in 1850 Kirkman went on to pose a more difficult but related problem. This problem, which appeared in "The Lady's and Gentleman's Diary" of 1850, has become to be known as Kirkman's Schoolgirl Problem and was presented as follows:

A teacher would like to take 15 schoolgirls out for a walk, the girls being arranged in 5 rows of three. The teacher would like to ensure equal chances of friendship between any two girls. Hence it is desirable to find different row arrangements for the 7 days of the week such that any pair of girls walk in the same row exactly one day of the week. 

This problem, which in general asks for a Steiner Triple System on 6t +3 varieties whose blocks can be partitioned into 3t + 1 sets so that any variety appears only once in a set, had not been settled in general until 1971 (Ray-Chaudhuri and Wilson).
\end{comment} 